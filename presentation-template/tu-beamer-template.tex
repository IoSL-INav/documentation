% Template for talks using the Corporate Design of the Freie Universitaet
%   Berlin, created following the guidelines on www.tu-berlin.de/cd by
%   Tobias G. Pfeiffer, <tobias.pfeiffer@math.tu-berlin.de>
% This file can be redistributed and/or modified in any way you like.
%   If you feel you have done significant improvements to this template,
%   please consider providing your modified version to
%   https://www.mi.tu-berlin.de/w/Mi/BeamerTemplateCorporateDesign

% altered by someone at TU. fiddled with and fixed some things by 
% Nicolas Werner (correct colours, no figure numbering, 
% some more stuff I can't remember)

\usepackage{amsmath,dsfont,listings}
\usetheme{Frankfurt}

\setbeamertemplate{sections/subsections in toc}[default]
\setbeamertemplate{items}[default]
\setbeamertemplate{itemize items}[default]
\setbeamertemplate{enumerate items}[default]


\hypersetup{colorlinks,citecolor={blue},linkcolor={},urlcolor={red}}

%%% TU logo
% small version for upper right corner of normal pages
\pgfdeclareimage[height=0.9cm]{university-logo}{TU_Logo_lang_RGB_rot}
\logo{\pgfuseimage{university-logo}}
% large version for upper right corner of title page
\pgfdeclareimage[height=1.085cm]{big-university-logo}{TU_Logo_lang_RGB_rot}
\newcommand{\titleimage}[1]{\pgfdeclareimage[height=2.92cm]{title-image}{#1}}
\titlegraphic{\pgfuseimage{title-image}}
%%% end TU logo




% NOTE: 1cm = 0.393 in = 28.346 pt;    1 pt = 1/72 in = 0.0352 cm
\setbeamersize{text margin right=3.5mm, text margin left=7.5mm}  % text margin

% colors to be used
\definecolor{text-grey}{rgb}{0.45, 0.45, 0.45} % grey text on white background
\definecolor{bg-grey}{rgb}{0.66, 0.65, 0.60} % grey background (for white text)
\definecolor{tu-blue}{RGB}{0, .2, .4} % blue text
\definecolor{tu-green}{RGB}{.6, .8, 0} % green text
%\definecolor{tu-red}{RGB}{204, 0, 0} % red text (used by \alert)
\definecolor{tu-red}{rgb}{.772, .055, .122}%{0.6,0,0} %% based on DARKRED TU-LOGO -- GIMP says TU DARKRED is RGB{153,0,0}

% switch off the sidebars
% TODO: loading \useoutertheme{sidebar} (which is maybe wanted) also inserts
%   a sidebar on title page (unwanted), also indents the page title (unwanted?),
%   and duplicates the navigation symbols (unwanted)
\setbeamersize{sidebar width left=0cm, sidebar width right=0mm}
\setbeamertemplate{sidebar right}{}
\setbeamertemplate{sidebar left}{}
%    XOR
% \useoutertheme{sidebar}

% frame title
% is truncated before logo and splits on two lines
% if neccessary (or manually using \\)
\setbeamertemplate{frametitle}{%
    \vskip-30pt \color{text-grey}\large%
    \begin{minipage}[b][23pt]{80.5mm}%
    \flushleft\insertframetitle%
    \end{minipage}%
}

%%% title page
% TODO: get rid of the navigation symbols on the title page.
%   actually, \frame[plain] *should* remove them...
\setbeamertemplate{title page}{
	
	% upper right: FU logo
	\hfill\pgfuseimage{big-university-logo} \\
	
	% title image of the presentation
	\begin{minipage}{11.6cm}
		\hspace{-1mm}\vspace{0.5cm}\inserttitlegraphic
	\end{minipage}

	% set the title and the author
	\parbox[top][1.35cm][c]{11cm}{\inserttitle}
	\vskip10pt
	\parbox[top][1.35cm][c]{11cm}{\color{text-grey} {\tiny \insertauthor\\ \insertinstitute\\ \insertdate}}
}
%%% end title page

%%% colors
\usecolortheme{lily}
\setbeamercolor*{normal text}{fg=black,bg=white}
\setbeamercolor*{alerted text}{fg=tu-red}
\setbeamercolor*{example text}{fg=tu-green}
\setbeamercolor*{structure}{fg=tu-blue}

\setbeamercolor*{block title}{fg=white,bg=black!50}
\setbeamercolor*{block title alerted}{fg=white,bg=black!50}
\setbeamercolor*{block title example}{fg=white,bg=black!50}

\setbeamercolor*{block body}{bg=black!10}
\setbeamercolor*{block body alerted}{bg=black!10}
\setbeamercolor*{block body example}{bg=black!10}

\setbeamercolor{bibliography entry author}{fg=tu-blue}
% TODO: this doesn't work at all:
\setbeamercolor{bibliography entry journal}{fg=text-grey}

\setbeamercolor{item}{fg=tu-blue}
%\setbeamercolor{navigation symbols}{fg=text-grey,bg=bg-grey}
\setbeamercolor{navigation symbols}{fg=text-grey,bg=bg-grey}
%%% end colors

%%% headline
\setbeamertemplate{headline}{
\vskip4pt\hfill\insertlogo\hspace{3.5mm} % logo on the right
\vskip6pt\color{tu-red}\rule{\textwidth}{0.4pt} % horizontal line was BLUE!
}

%%% footline
\setbeamercolor{upper separation line head}{bg=red}
\setbeamercolor{section in head/foot}{fg=black, bg=white}
\setbeamercolor{frametitle}{fg=red, bg=white}

\newcommand{\footlinetext}{
	\insertshortinstitute, \insertshorttitle, \insertshortdate
}


\makeatletter
%THIS IS OLD
%\setbeamertemplate{footline}{
%\vskip5pt\color{tu-red}\rule{\textwidth}{0.4pt}\\ % horizontal line
%\vskip2pt
%\makebox[123mm]{\hspace{7.5mm}
%\color{black}\footlinetext
%\hfill \raisebox{-1pt}{\phantom{\usebeamertemplate***{navigation symbols}}}
%%\hfill \raisebox{-1pt}{\phantom{\big{lol}}}%\raisebox{-1pt}{\usebeamertemplate***{navigation symbols}}
%\hfill \insertframenumber}
%\vskip4pt
%}
\setbeamertemplate{footline}{
\vskip5pt\color{tu-red}\rule{\textwidth}{0.4pt}\\ % horizontal line
\pgfuseshading{beamer@barshade}%
  \ifbeamer@sb@subsection%
    \vskip-9.75ex%
  \else%
    \vskip-7ex%
  \fi%
  % fügt die Navigationsleiste ein:
  \begin{beamercolorbox}[ignorebg,dp=3.75ex,ht=2.25ex]{subsection in head/foot}
	\insertnavigation{0.5\paperwidth} % <======= Added 0.5 here
  \end{beamercolorbox}%
  % kein Plan was das macht :D
  \ifbeamer@sb@subsection%
    \begin{beamercolorbox}[ignorebg,ht=2.125ex,dp=1.125ex,%
      leftskip=.3cm,rightskip=.3cm plus1fil]{subsection in head/foot}
      \usebeamerfont{subsection in head/foot}\insertsubsectionhead
    \end{beamercolorbox}%
  \fi%
  %
  % GROßE Trennlinie:
  %\begin{beamercolorbox}[colsep=1.5pt,ht=.75ex]{upper separation line head}
  %\end{beamercolorbox}
  \vskip-2ex%
  \hskip0.98\textwidth \insertframenumber % hfill
  \vskip4pt % vom unteren Rand 4 pt weg
}

\makeatother
%%% end footline

%%% settings for listings package
\lstset{extendedchars=true, showstringspaces=false, basicstyle=\footnotesize\sffamily, tabsize=2, breaklines=true, breakindent=10pt, frame=l, columns=fullflexible}
\lstset{language=Java} % this sets the syntax highlighting
\lstset{mathescape=true} % this switches on $...$ substitution in code
% enables UTF-8 in source code:
\lstset{literate={ä}{{\"a}}1 {ö}{{\"o}}1 {ü}{{\"u}}1 {Ä}{{\"A}}1 {Ö}{{\"O}}1 {Ü}{{\"U}}1 {ß}{\ss}1}


%%% end listings
