% Quelle: http://tex.stackexchange.com/questions/56417/list-of-figures-beamer
\usepackage{ifthen}
\usepackage{xstring}

\makeatletter

%\defbeamertemplate*{caption label separator}{colon}{:}

\def\dotfill{%
  \leavevmode
  \cleaders \hb@xt@ .44em{\hss.\hss}\hfill
  \kern\z@}

\newcommand{\myVL}[2]{\tmpVL{#1}{#2}}

\newcommand{\tmpVL}[2]{Page #1, last accessed: #2}

\newcommand{\myhref}[2]{\tmphref{#1}{#2}}

\newcommand{\tmphref}[2]{\href{#1}{Source}, last accessed: #2}

\AtEndDocument{%
  % sorgt dafür, dass die Dateien 
  % *.lof -> list of figures
  % *.lot -> list of tables
  % geeleert werden oder ggf. neu erzeugt werden (erzwungen)
  \clearpage
  \beamer@tempcount=\c@page\advance\beamer@tempcount by -1%
  \if@filesw
  \newwrite\tf@lof
  \immediate\openout\tf@lof\jobname.lof\relax
  \newwrite\tf@lot
  \immediate\openout\tf@lot\jobname.lot\relax
  \fi
}

% beamer caption ändern....
\long\def\beamer@makecaption#1#2#3#4{%
  % [ von caption selbst]
  % #1 == Typ == Bild / Tabelle....
  % #2 == optionale Beschreibung
  % #3 == caption Beschreibung
  % bsp: \caption[optionale Beschreibung]{caption Beschreibung}

  % calls: \beamer@makecaption{#1}{\ignorespaces #3}{yes/no}{\ignorespaces #2}  

  % ----------------------------------------------

  % #1 == Typ == Tabelle / Bild.....
  % #2 == Captionbeschreibung
  % #3 == yes/no if you want complete Text under picture
  % #4 == optionale Beschreibung vom Bild
  \hypertarget{\insertframenumber}{}{}
  % \def\insertcaptionname{\csname#1name\endcsname}%
  % liefert Figure, weil babel auf englisch....
  \def\insertcaptionname{Fig.}%
  \def\insertcaptionnumber{\csname the#1\endcsname}%
  \edef\insertframenumber{\theframenumber}%
%  \ifthenelse{\equal{#3}{\empty}}{%  
    \def\insertlistcaption{#2}%
%  }{%
%    \def\insertlistcaption{#3}%
%  }
  	\def\insertsource{#4}%
    %
    \def\insertcaption{#2}%
    \ifthenelse{\equal{#1}{figure}}{%  
      \addtocontents{lof}{\relax\protect\listoffigureformat{\insertcaptionnumber}{\insertlistcaption}{\protect\hyperlink{\insertframenumber}{\insertframenumber}}{\insertsource}}{}{}%
      }{}
    \ifthenelse{\equal{#1}{table}}{%  
      \addtocontents{lot}{\protect\listoftableformat{\insertcaptionnumber}{\insertlistcaption}{\insertframenumber}}{}{}%
      }{}
  \ifthenelse{\equal{#3}{no}}
  { 
  	% Text beim Bild
    %\addtocontents{lof}{#4,jaaaaaaaaaaaaaaaaaaaaaaa}%
  	\nobreak\vskip\abovecaptionskip\nobreak
  	\sbox\@tempboxa{
  		%\usebeamertemplate**{caption}
  		%\raggedright
    	{%	
      		\usebeamercolor[fg]{caption name}%
      		\usebeamerfont*{caption name}%
      		\insertcaptionname~\insertcaptionnumber
    	}
    	\par
  	}%
  	\ifdim \wd\@tempboxa >\hsize
    	%\usebeamertemplate**{caption}\par
    	%\raggedright
    	{%
      		\usebeamercolor[fg]{caption name}%
      		\usebeamerfont*{caption name}%
      		\insertcaptionname~\insertcaptionnumber
    	}
    	\par
 	\else
  		\global \@minipagefalse
  		\hb@xt@\hsize{\hfil\box\@tempboxa\hfil}%
  	\fi
  	\nobreak\vskip\belowcaptionskip\nobreak%
  }{ 
	% Text beim Bild
    %\addtocontents{lof}{#4,jaaaaaaaaaaaaaaaaaaaaaaa}%
  	\nobreak\vskip\abovecaptionskip\nobreak
  	\sbox\@tempboxa{
  		%\usebeamertemplate**{caption}
  		%\raggedright
    	{%	
% % nw: figure numbering in slides
%      		\usebeamercolor[fg]{caption name}%
%      		\usebeamerfont*{caption name}%
%      		\insertcaptionname~\insertcaptionnumber
%      		\usebeamertemplate{caption label separator}%
    	}
    	\insertcaption\par
  	}%
  	\ifdim \wd\@tempboxa >\hsize
    	%\usebeamertemplate**{caption}\par
    	%\raggedright
    	{%
      		\usebeamercolor[fg]{caption name}%
      		\usebeamerfont*{caption name}%
      		\insertcaptionname~\insertcaptionnumber
      		\usebeamertemplate{caption label separator}%
    	}
    	\insertcaption\par
 	\else
  		\global \@minipagefalse
  		\hb@xt@\hsize{\hfil\box\@tempboxa\hfil}%
  	\fi
  	\nobreak\vskip\belowcaptionskip\nobreak%  }
  }
}

\def\listoffiguresectionformat#1#2{%
  % \listoffiguresectionformat{\insertsectionhead}{\insertframenumber}
  % #1 == Sectiontitel
  % #2 == Seite
  \setlength{\leftskip}{2ex}
  \setlength{\rightskip}{-0.6ex}
  \setlength{\parindent}{-3ex}
  %
  {\usebeamercolor[fg]{bibliography entry author} #1}%
  \dotfill%
  \hspace*{0.6ex}%
  \makebox[3ex][r]{#2}\par%
  %
  \setlength{\leftskip}{3ex}
  \setlength{\rightskip}{0ex}
  \setlength{\parindent}{-3ex}
}

\newcounter{tmpImageCounter}
\setcounter{tmpImageCounter}{0}

\def\listoffigureformat#1#2#3#4{%
	\ifnum\value{tmpImageCounter}=#1
		\typeout{Counter \thetmpImageCounter  ist gleich #1. Das Muss das erste Bild sein...}
		% \listoffigureformat{\insertcaptionnumber}{\insertlistcaption}{\insertframenumber}{\insertsource}
		% #1 == Bildnummer
		% #2 == Captionbeschreibung
		% #3 == Seite, wo das Bild ist
		% #4 == optionale Beschreibung vom Bild 
		% \caption[optionale Beschreibung]{Captionbeschreibung}
		\makebox[2ex][r]{#1}%
		\hspace{1ex}%
		{\usebeamercolor[fg]{bibliography entry author} #2}%
		\ifthenelse{\equal{#4}{\empty}}{}{ -- #4}%
		\dotfill%
		\makebox[3ex][r]{#3}\par%
		\refstepcounter{tmpImageCounter}
	% Bild wurde schonmal gesetzt....nee nicht nochmal...
  	\else
		\PackageWarning{listoffigureformat}{Bild wurde doppelt gesetzt -- wird ignoriert!}
		\typeout{Counter \thetmpImageCounter ist groesser oder kleiner #1.}
	\fi

}

% der eigentliche Befehl, der am Ende die Liste der Figures anzeigt
\def\listoffigures{%
  \setlength{\leftskip}{3ex}
  \setlength{\parindent}{-3ex}
  \@starttoc{lof}%
}

\def\listoftableformat#1#2#3{%
 % \listoftableformat{\insertcaptionnumber}{\insertlistcaption}{\insertframenumber}
 % #1 == Tabellennummer
 % #2 == Captionbeschreibung 
 % #3 == Seite, wo die Tabelle ist
 \makebox[2ex][r]{#1}\hspace{1ex}#2\dotfill\makebox[2ex][r]{#3}\par
}

\def\listoftables{%
  \setlength{\leftskip}{3ex}
  \setlength{\parindent}{-3ex}
  \@starttoc{lot}%
}

% den eigentlichen Caption Befehl umschreiben
\long\def\@caption#1[#2]#3{
  % #1 == Typ == Bild / Tabelle....
  % #2 == optionale Beschreibung
  % #3 == caption Beschreibung
  % bsp: \caption[optionale Beschreibung]{caption Beschreibung}
  \par\nobreak
  \begingroup
    \@parboxrestore
    \if@minipage
      \@setminipage
    \fi
    \beamer@makecaption{#1}{\ignorespaces #3}{yes}{\ignorespaces  #2}\par\nobreak
    \endgroup
}

\def\nocaption{\refstepcounter\@captype\@dblarg{\@nocaption\@captype}}

\long\def\@nocaption#1[#2]#3{
  % #1 == Typ == Bild / Tabelle....
  % #2 == optionale Beschreibung
  % #3 == caption Beschreibung
  % bsp: \caption[optionale Beschreibung]{caption Beschreibung}
  \par\nobreak
  \begingroup
    \@parboxrestore
    \if@minipage
      \@setminipage
    \fi
    \beamer@makecaption{#1}{\ignorespaces #3}{no}{\ignorespaces #2}\par\nobreak
    \endgroup
}



\makeatother
